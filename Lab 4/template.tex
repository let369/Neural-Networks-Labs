\documentclass{article}
\usepackage[
        a4paper,% other options: a3paper, a5paper, etc
        left=3cm,
        right=3cm,
        top=3cm,
        bottom=4cm,
        % use vmargin=2cm to make vertical margins equal to 2cm.
        % us  hmargin=3cm to make horizontal margins equal to 3cm.
        % use margin=3cm to make all margins  equal to 3cm.
]{geometry}
%\usepackage[utf8x]{inputenc}
\usepackage{graphicx}
\usepackage{caption}
\usepackage{enumerate}
\usepackage{subcaption}
\usepackage[procnames]{listings}
\usepackage{color}
\usepackage{amssymb}
\usepackage{amsmath}      
\usepackage{comment}
\usepackage{hyperref}
\usepackage{blindtext}
\usepackage[scaled=.8]{sourcecodepro}
\setcounter{section}{+1}
\definecolor{codegreen}{rgb}{0,0.6,0}
\definecolor{codegray}{rgb}{0.5,0.5,0.5}
\definecolor{codepurple}{rgb}{0.58,0,0.82}
\definecolor{backcolour}{rgb}{0.95,0.95,0.92}
 
%\renewcommand*\ttdefault{pcr} 
 
\lstdefinestyle{mystyle}{
    backgroundcolor=\color{backcolour},   
    commentstyle=\color{codegreen},
    keywordstyle=\color{blue},
    numberstyle=\tiny\color{codegray},
    stringstyle=\color{codepurple},
    basicstyle=\ttfamily,
    breakatwhitespace=false,         
    breaklines=true,                 
    captionpos=b,                    
    keepspaces=true,                 
    numbers=left,                    
    numbersep=5pt,                  
    showspaces=false,                
    showstringspaces=false,
    showtabs=false,                  
    tabsize=2
}

\title{Lab Assignment 4\\ {\Large Convolutional Neural Networks}}
\date{\today}
\author{
	Panagiotis Tzafos (s3302148)\\
 	Eleftherios Karamoulas(s)// 
}

\lstset{style=mystyle, language=Matlab}
\begin{document}
\maketitle
\section{Theory questions}
\begin{enumerate}
\item The pooling layer serves to progressively reduce the spatial size of the representation, to reduce the number of parameters and amount of computation in the network, and hence to also control overfitting(CS231n lecture). In practice, its use is to downsample its input, reducing the amount of resources that the network needs in order to perform the computations by reducing the dimensions of the input using a filter, a stride and an elementwise activation function. Also, after the downsampling it is obvious that the network will have less parameters. As a result, it will be able to generalize better in new situations and avoid overfitting(small training set error, large error for new examples). 
\item
\item
\end{enumerate}


\end{document}
